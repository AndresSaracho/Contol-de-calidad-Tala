\documentclass[]{article}
\usepackage{lmodern}
\usepackage{amssymb,amsmath}
\usepackage{ifxetex,ifluatex}
\usepackage{fixltx2e} % provides \textsubscript
\ifnum 0\ifxetex 1\fi\ifluatex 1\fi=0 % if pdftex
  \usepackage[T1]{fontenc}
  \usepackage[utf8]{inputenc}
\else % if luatex or xelatex
  \ifxetex
    \usepackage{mathspec}
  \else
    \usepackage{fontspec}
  \fi
  \defaultfontfeatures{Ligatures=TeX,Scale=MatchLowercase}
\fi
% use upquote if available, for straight quotes in verbatim environments
\IfFileExists{upquote.sty}{\usepackage{upquote}}{}
% use microtype if available
\IfFileExists{microtype.sty}{%
\usepackage{microtype}
\UseMicrotypeSet[protrusion]{basicmath} % disable protrusion for tt fonts
}{}
\usepackage[margin=1in]{geometry}
\usepackage{hyperref}
\hypersetup{unicode=true,
            pdftitle={Algunos herramientas gráficas para verificar la precipitación},
            pdfauthor={Rafael Navas},
            pdfborder={0 0 0},
            breaklinks=true}
\urlstyle{same}  % don't use monospace font for urls
\usepackage{color}
\usepackage{fancyvrb}
\newcommand{\VerbBar}{|}
\newcommand{\VERB}{\Verb[commandchars=\\\{\}]}
\DefineVerbatimEnvironment{Highlighting}{Verbatim}{commandchars=\\\{\}}
% Add ',fontsize=\small' for more characters per line
\usepackage{framed}
\definecolor{shadecolor}{RGB}{248,248,248}
\newenvironment{Shaded}{\begin{snugshade}}{\end{snugshade}}
\newcommand{\AlertTok}[1]{\textcolor[rgb]{0.94,0.16,0.16}{#1}}
\newcommand{\AnnotationTok}[1]{\textcolor[rgb]{0.56,0.35,0.01}{\textbf{\textit{#1}}}}
\newcommand{\AttributeTok}[1]{\textcolor[rgb]{0.77,0.63,0.00}{#1}}
\newcommand{\BaseNTok}[1]{\textcolor[rgb]{0.00,0.00,0.81}{#1}}
\newcommand{\BuiltInTok}[1]{#1}
\newcommand{\CharTok}[1]{\textcolor[rgb]{0.31,0.60,0.02}{#1}}
\newcommand{\CommentTok}[1]{\textcolor[rgb]{0.56,0.35,0.01}{\textit{#1}}}
\newcommand{\CommentVarTok}[1]{\textcolor[rgb]{0.56,0.35,0.01}{\textbf{\textit{#1}}}}
\newcommand{\ConstantTok}[1]{\textcolor[rgb]{0.00,0.00,0.00}{#1}}
\newcommand{\ControlFlowTok}[1]{\textcolor[rgb]{0.13,0.29,0.53}{\textbf{#1}}}
\newcommand{\DataTypeTok}[1]{\textcolor[rgb]{0.13,0.29,0.53}{#1}}
\newcommand{\DecValTok}[1]{\textcolor[rgb]{0.00,0.00,0.81}{#1}}
\newcommand{\DocumentationTok}[1]{\textcolor[rgb]{0.56,0.35,0.01}{\textbf{\textit{#1}}}}
\newcommand{\ErrorTok}[1]{\textcolor[rgb]{0.64,0.00,0.00}{\textbf{#1}}}
\newcommand{\ExtensionTok}[1]{#1}
\newcommand{\FloatTok}[1]{\textcolor[rgb]{0.00,0.00,0.81}{#1}}
\newcommand{\FunctionTok}[1]{\textcolor[rgb]{0.00,0.00,0.00}{#1}}
\newcommand{\ImportTok}[1]{#1}
\newcommand{\InformationTok}[1]{\textcolor[rgb]{0.56,0.35,0.01}{\textbf{\textit{#1}}}}
\newcommand{\KeywordTok}[1]{\textcolor[rgb]{0.13,0.29,0.53}{\textbf{#1}}}
\newcommand{\NormalTok}[1]{#1}
\newcommand{\OperatorTok}[1]{\textcolor[rgb]{0.81,0.36,0.00}{\textbf{#1}}}
\newcommand{\OtherTok}[1]{\textcolor[rgb]{0.56,0.35,0.01}{#1}}
\newcommand{\PreprocessorTok}[1]{\textcolor[rgb]{0.56,0.35,0.01}{\textit{#1}}}
\newcommand{\RegionMarkerTok}[1]{#1}
\newcommand{\SpecialCharTok}[1]{\textcolor[rgb]{0.00,0.00,0.00}{#1}}
\newcommand{\SpecialStringTok}[1]{\textcolor[rgb]{0.31,0.60,0.02}{#1}}
\newcommand{\StringTok}[1]{\textcolor[rgb]{0.31,0.60,0.02}{#1}}
\newcommand{\VariableTok}[1]{\textcolor[rgb]{0.00,0.00,0.00}{#1}}
\newcommand{\VerbatimStringTok}[1]{\textcolor[rgb]{0.31,0.60,0.02}{#1}}
\newcommand{\WarningTok}[1]{\textcolor[rgb]{0.56,0.35,0.01}{\textbf{\textit{#1}}}}
\usepackage{graphicx,grffile}
\makeatletter
\def\maxwidth{\ifdim\Gin@nat@width>\linewidth\linewidth\else\Gin@nat@width\fi}
\def\maxheight{\ifdim\Gin@nat@height>\textheight\textheight\else\Gin@nat@height\fi}
\makeatother
% Scale images if necessary, so that they will not overflow the page
% margins by default, and it is still possible to overwrite the defaults
% using explicit options in \includegraphics[width, height, ...]{}
\setkeys{Gin}{width=\maxwidth,height=\maxheight,keepaspectratio}
\IfFileExists{parskip.sty}{%
\usepackage{parskip}
}{% else
\setlength{\parindent}{0pt}
\setlength{\parskip}{6pt plus 2pt minus 1pt}
}
\setlength{\emergencystretch}{3em}  % prevent overfull lines
\providecommand{\tightlist}{%
  \setlength{\itemsep}{0pt}\setlength{\parskip}{0pt}}
\setcounter{secnumdepth}{0}
% Redefines (sub)paragraphs to behave more like sections
\ifx\paragraph\undefined\else
\let\oldparagraph\paragraph
\renewcommand{\paragraph}[1]{\oldparagraph{#1}\mbox{}}
\fi
\ifx\subparagraph\undefined\else
\let\oldsubparagraph\subparagraph
\renewcommand{\subparagraph}[1]{\oldsubparagraph{#1}\mbox{}}
\fi

%%% Use protect on footnotes to avoid problems with footnotes in titles
\let\rmarkdownfootnote\footnote%
\def\footnote{\protect\rmarkdownfootnote}

%%% Change title format to be more compact
\usepackage{titling}

% Create subtitle command for use in maketitle
\providecommand{\subtitle}[1]{
  \posttitle{
    \begin{center}\large#1\end{center}
    }
}

\setlength{\droptitle}{-2em}

  \title{Algunos herramientas gráficas para verificar la precipitación}
    \pretitle{\vspace{\droptitle}\centering\huge}
  \posttitle{\par}
    \author{Rafael Navas}
    \preauthor{\centering\large\emph}
  \postauthor{\par}
      \predate{\centering\large\emph}
  \postdate{\par}
    \date{26/9/2019}


\begin{document}
\maketitle

\hypertarget{introduccion}{%
\section{Introducción}\label{introduccion}}

Este documento muestra algunas herrameintas gráficas en R para verificar
los datos de precipitación. Trabaja con el script creado por Andres
``precipitacion\_analisis\_mensual.R'' localizado en la carpeta raiz del
proyecto
(\url{https://github.com/AndresSaracho/Contol-de-calidad-Tala}).

\hypertarget{ploteo-de-estaciones}{%
\section{Ploteo de estaciones}\label{ploteo-de-estaciones}}

Esta sección muestra como plotear las estaciones desde R. Primero
cargamos las librerias y los shapefiles. Las variables utm es el sistema
de referencia (UTM 21S).

\begin{Shaded}
\begin{Highlighting}[]
\KeywordTok{library}\NormalTok{(maptools)}
\NormalTok{utm =}\StringTok{ }\KeywordTok{CRS}\NormalTok{(}\StringTok{"+proj=utm +zone=21 +south +ellps=WGS84 +datum=WGS84 +units=m +no_defs"}\NormalTok{)}
\NormalTok{pluvio_shp =}\StringTok{ }\KeywordTok{readShapePoints}\NormalTok{(}\StringTok{"./Datos/Estaciones.shp"}\NormalTok{, }\DataTypeTok{proj4string=}\NormalTok{utm)}
\NormalTok{pluvio_shp}\OperatorTok{$}\NormalTok{Numero[}\DecValTok{9}\OperatorTok{:}\DecValTok{10}\NormalTok{] =}\StringTok{ }\KeywordTok{c}\NormalTok{(}\StringTok{"INIA"}\NormalTok{, }\StringTok{"Junco"}\NormalTok{)}
\NormalTok{basin_shp =}\StringTok{ }\KeywordTok{readShapePoly}\NormalTok{(}\StringTok{"./Datos/Tala.shp"}\NormalTok{, }\DataTypeTok{force_ring =}\NormalTok{ T, }\DataTypeTok{proj4string=}\NormalTok{utm)}
\end{Highlighting}
\end{Shaded}

Ahora podemos plotear de la siguiente manera

\begin{Shaded}
\begin{Highlighting}[]
\KeywordTok{plot}\NormalTok{(pluvio_shp,}\DataTypeTok{axes=}\NormalTok{T)}
\KeywordTok{plot}\NormalTok{(basin_shp,}\DataTypeTok{add=}\NormalTok{T)}
\KeywordTok{text}\NormalTok{(}\KeywordTok{coordinates}\NormalTok{(pluvio_shp)[}\OperatorTok{-}\DecValTok{8}\NormalTok{,], pluvio_shp}\OperatorTok{$}\NormalTok{Numero[}\OperatorTok{-}\DecValTok{8}\NormalTok{], }\DataTypeTok{pos=}\DecValTok{2}\NormalTok{)}
\KeywordTok{text}\NormalTok{(}\KeywordTok{coordinates}\NormalTok{(pluvio_shp)[}\DecValTok{8}\NormalTok{,}\DecValTok{1}\NormalTok{],}\KeywordTok{coordinates}\NormalTok{(pluvio_shp)[}\DecValTok{8}\NormalTok{,}\DecValTok{2}\NormalTok{],}
\NormalTok{     pluvio_shp}\OperatorTok{$}\NormalTok{Numero[}\DecValTok{8}\NormalTok{], }\DataTypeTok{pos=}\DecValTok{4}\NormalTok{, }\DataTypeTok{col=}\DecValTok{2}\NormalTok{)}
\end{Highlighting}
\end{Shaded}

\includegraphics{Precipitacion_HerramientasGraficas_files/figure-latex/unnamed-chunk-2-1.pdf}

Vemos que la estación 1257 esta en dos sitios. Probablemente fue movida
y los registros pueden o no solaparse.

También podemos cambiar de sistema de coordenadas con la funcion
spTransform() y definiendo nuevos sistemas de coordenadas.

Por ejemplo en kilometros:

\begin{Shaded}
\begin{Highlighting}[]
\NormalTok{utkm =}\StringTok{ }\KeywordTok{CRS}\NormalTok{(}\StringTok{"+proj=utm +zone=21 +south +ellps=WGS84 +datum=WGS84 +units=km +no_defs"}\NormalTok{)}
\NormalTok{pluvio_shp_km =}\StringTok{ }\KeywordTok{spTransform}\NormalTok{(pluvio_shp, utkm)}
\NormalTok{basin_shp_km =}\StringTok{ }\KeywordTok{spTransform}\NormalTok{(basin_shp, utkm)}
\KeywordTok{plot}\NormalTok{(pluvio_shp_km,}\DataTypeTok{axes=}\NormalTok{T, }\DataTypeTok{xlab=}\StringTok{"Long (UTM 21S, km)"}\NormalTok{, }\DataTypeTok{ylab=}\StringTok{"Lat (UTM 21S, km)"}\NormalTok{)}
\KeywordTok{plot}\NormalTok{(basin_shp_km,}\DataTypeTok{add=}\NormalTok{T)}
\end{Highlighting}
\end{Shaded}

\includegraphics{Precipitacion_HerramientasGraficas_files/figure-latex/unnamed-chunk-3-1.pdf}

otra alternativa en wgs84 en grados:

\begin{Shaded}
\begin{Highlighting}[]
\NormalTok{dec =}\StringTok{ }\KeywordTok{CRS}\NormalTok{(}\StringTok{"+init=epsg:4326"}\NormalTok{)}
\NormalTok{pluvio_shp_dec =}\StringTok{ }\KeywordTok{spTransform}\NormalTok{(pluvio_shp, dec)}
\NormalTok{basin_shp_dec =}\StringTok{ }\KeywordTok{spTransform}\NormalTok{(basin_shp, dec)}
\KeywordTok{plot}\NormalTok{(pluvio_shp_dec,}\DataTypeTok{axes=}\NormalTok{T, }\DataTypeTok{xlab=}\StringTok{"Long (WGS84, º)"}\NormalTok{, }\DataTypeTok{ylab=}\StringTok{"Lat (WGS84, º)"}\NormalTok{)}
\KeywordTok{plot}\NormalTok{(basin_shp_dec,}\DataTypeTok{add=}\NormalTok{T)}
\end{Highlighting}
\end{Shaded}

\includegraphics{Precipitacion_HerramientasGraficas_files/figure-latex/unnamed-chunk-4-1.pdf}

\hypertarget{lectura-de-los-datos-de-precipitacion}{%
\section{Lectura de los datos de
precipitación}\label{lectura-de-los-datos-de-precipitacion}}

Los datos de precipitación han sido guardados en la carpeta
Datos/precipitacion\_diaria.RDS. Han sido calculados con el script
"precipitacion\_analisis\_menusal.R (version del 26/09/2019). Para leer
dichos archivos se procede de la siguiente manera:

\begin{Shaded}
\begin{Highlighting}[]
\KeywordTok{library}\NormalTok{(zoo)}
\end{Highlighting}
\end{Shaded}

\begin{verbatim}
## 
## Attaching package: 'zoo'
\end{verbatim}

\begin{verbatim}
## The following objects are masked from 'package:base':
## 
##     as.Date, as.Date.numeric
\end{verbatim}

\begin{Shaded}
\begin{Highlighting}[]
\NormalTok{Pdata =}\StringTok{ }\KeywordTok{readRDS}\NormalTok{(}\StringTok{"./Datos/precipitacion_diaria.RDS"}\NormalTok{)}
\end{Highlighting}
\end{Shaded}

Estos datos estan organizados en una matriz zoo de cinco columnas

\begin{Shaded}
\begin{Highlighting}[]
\KeywordTok{head}\NormalTok{(Pdata)}
\end{Highlighting}
\end{Shaded}

\begin{verbatim}
##            p1257_z p1176_z p1232_z pINIA_z pjunco_z
## 2005-12-09       0      NA      NA     7.3      0.2
## 2005-12-10       0      NA      NA     0.0      0.0
## 2005-12-11       0      NA      NA     0.0      0.0
## 2005-12-12       0      NA      NA     0.0     14.0
## 2005-12-13       0      NA      NA     0.0      0.0
## 2005-12-14       0      NA      NA     0.0      0.0
\end{verbatim}

Podemos solicitar un resumen

\begin{Shaded}
\begin{Highlighting}[]
\KeywordTok{summary}\NormalTok{(Pdata)}
\end{Highlighting}
\end{Shaded}

\begin{verbatim}
##      Index               p1257_z           p1176_z       
##  Min.   :2005-12-09   Min.   :  0.000   Min.   :  0.000  
##  1st Qu.:2009-04-13   1st Qu.:  0.000   1st Qu.:  1.000  
##  Median :2012-08-16   Median :  0.000   Median :  2.000  
##  Mean   :2012-08-16   Mean   :  4.126   Mean   :  5.965  
##  3rd Qu.:2015-12-20   3rd Qu.:  2.000   3rd Qu.:  2.000  
##  Max.   :2019-04-25   Max.   :140.000   Max.   :168.000  
##                       NA's   :308       NA's   :3242     
##     p1232_z           pINIA_z           pjunco_z      
##  Min.   :  0.000   Min.   :  0.000   Min.   :  0.000  
##  1st Qu.:  2.000   1st Qu.:  0.000   1st Qu.:  0.000  
##  Median :  2.000   Median :  0.000   Median :  0.000  
##  Mean   :  8.835   Mean   :  3.816   Mean   :  3.791  
##  3rd Qu.:  5.000   3rd Qu.:  0.000   3rd Qu.:  0.400  
##  Max.   :122.000   Max.   :190.800   Max.   :142.000  
##  NA's   :3134                        NA's   :212
\end{verbatim}

Vemos que la estación del INIA no tiene datos faltantes. Por otro lado,
las estaciones 1176 y 1232 tienen mas de ocho años de datos faltantes.
Para visualizar podemos graficar de la siguiente manera:

\begin{Shaded}
\begin{Highlighting}[]
\KeywordTok{plot}\NormalTok{(Pdata)}
\end{Highlighting}
\end{Shaded}

\includegraphics{Precipitacion_HerramientasGraficas_files/figure-latex/unnamed-chunk-8-1.pdf}

\hypertarget{curva-de-doble-masa}{%
\section{Curva de doble masa}\label{curva-de-doble-masa}}

La curva de doble masa compara la precipitacion acumulada de una
estacion de referencia con otra de la cual tenemos dudas. En este caso
vamos a comparar la estacion del INIA con la estación 1257.

\begin{Shaded}
\begin{Highlighting}[]
\CommentTok{# La funcion cumsum no puede manejar valores NA, entonces cambiamos valores NA por cero}
\NormalTok{xcum =}\StringTok{ }\NormalTok{Pdata}\OperatorTok{$}\NormalTok{pINIA_z}
\NormalTok{ycum =}\StringTok{ }\NormalTok{Pdata}\OperatorTok{$}\NormalTok{p1257_z}
\NormalTok{xcum[}\KeywordTok{is.na}\NormalTok{(xcum)] =}\StringTok{ }\DecValTok{0}
\NormalTok{ycum[}\KeywordTok{is.na}\NormalTok{(ycum)] =}\StringTok{ }\DecValTok{0}

\CommentTok{# hacemos la suma}
\NormalTok{xcum =}\StringTok{ }\KeywordTok{cumsum}\NormalTok{(xcum)}
\NormalTok{ycum =}\StringTok{ }\KeywordTok{cumsum}\NormalTok{(ycum)}
\end{Highlighting}
\end{Shaded}

Para graficar procedemos como sigue:

\begin{Shaded}
\begin{Highlighting}[]
\KeywordTok{plot}\NormalTok{(xcum, ycum,}
     \DataTypeTok{xlab=}\StringTok{"Acumulado INIA (mm)"}\NormalTok{, }\DataTypeTok{ylab=}\StringTok{"Acumulado 1257 (mm)"}\NormalTok{, }
     \DataTypeTok{xlim=}\KeywordTok{c}\NormalTok{(}\DecValTok{0}\NormalTok{, }\KeywordTok{max}\NormalTok{(xcum,ycum)), }\DataTypeTok{ylim=}\KeywordTok{c}\NormalTok{(}\DecValTok{0}\NormalTok{, }\KeywordTok{max}\NormalTok{(xcum,ycum)))}
\KeywordTok{abline}\NormalTok{(}\DataTypeTok{a=}\DecValTok{0}\NormalTok{, }\DataTypeTok{b=}\DecValTok{1}\NormalTok{)}

\KeywordTok{par}\NormalTok{(}\DataTypeTok{new =}\NormalTok{ T)}
\KeywordTok{plot}\NormalTok{(}\KeywordTok{index}\NormalTok{(Pdata), xcum, }\DataTypeTok{type=}\StringTok{"n"}\NormalTok{, }\DataTypeTok{xlab=}\StringTok{""}\NormalTok{, }\DataTypeTok{ylab=}\StringTok{""}\NormalTok{, }\DataTypeTok{xaxt=}\StringTok{"n"}\NormalTok{, }\DataTypeTok{yaxt=}\StringTok{"n"}\NormalTok{)}
\KeywordTok{axis}\NormalTok{(}\DecValTok{3}\NormalTok{, }\DataTypeTok{at=}\KeywordTok{index}\NormalTok{(Pdata), }\DataTypeTok{labels =} \KeywordTok{format}\NormalTok{(}\KeywordTok{index}\NormalTok{(Pdata), }\StringTok{"%Y"}\NormalTok{), }\DataTypeTok{tck=}\DecValTok{0}\NormalTok{)}
\end{Highlighting}
\end{Shaded}

\includegraphics{Precipitacion_HerramientasGraficas_files/figure-latex/unnamed-chunk-10-1.pdf}

Los acumulados son similares, no obstante el perido de 2005-2012 la
estacion de INIA mide mas lluvia que la 1257. La diferencia es
aproximadamente 15-20\%. De 2012 a 2016 ambas estaciones parecen medir
igual (la pendiente del periodo es 1). Luego, del periodo 2016 hasta el
presente la estación 1257 mide mas lluvia que la estacion del INIA.


\end{document}
